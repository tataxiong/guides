\documentclass{article}
\usepackage{listings,framed,xcolor,xeCJK,graphicx,float,amsmath}
\usepackage[linkcolor=red,pdfborder={0 0 0},CJKbookmarks=true]{hyperref}
\setmainfont[BoldFont=SimHei]{SimSun}
\definecolor{lgray}{rgb}{0.95,0.95,0.95}
\lstset{
		basicstyle=\tt,
%		numbers=left,
%		stepnumber=1,
		frame=L,
		columns=fullflexible,
		breaklines=true,
		keywordstyle=\color{blue},
		backgroundcolor=\color{lgray},
		stringstyle=\color{red}
		boxpos=t
		}

\title{A Guide to Git and Github}
\author{Internet Service Department}
\date{\today}
\begin{document}
	\maketitle
	\section{安装与配置} % (fold)
	\label{sec:安装与配置}
		\begin{enumerate}
			\item 首先,注册一个\href{http://github.com/}{Github}帐号。
			\\在\href{http://msysgit.github.io/}{此处}下载并安装Git for Windows,通常可以直接使用默认配置。安装好之后可以在「开始」菜单中找到「Git Bash」和「Git GUI」两个程序。
			\item 打开「Git Bash」,首先设置用户名和邮箱,在命令行中键入如下命令:
			\begin{quote}
				\begin{lstlisting}
$ git config --global user.name "Your Name"
$ git config --global user.email yourmail@server.com
				\end{lstlisting}
			\end{quote}
			\item 然后创建SSH密钥
			\begin{quote}
				\begin{lstlisting}
$ ssh-keygen -C 'yourmail@server.com' -t rsa
				\end{lstlisting}
			\end{quote}
			\item Git Bash会询问储存密钥的路径,方便起见可以使用其默认路径,在windows下即为用户文件夹。生成成功后可以在之前设定的路径中找到{\tt id\_rsa.pub}文件,用记事本打开,文件中的字符即为所需的SSH密钥。
			\item 登录Github首页,点击Account Setting $\to$ SSH keys $\to$ Add SSH key。title可以随便填写,以可以判断这是自己电脑为原则。在key的输入框中粘贴{\tt id\_rsa.pub}文件中的内容,点击Apply即可。
			\\测试与 GitHub 是否连接成功:
			\begin{quote}
				\begin{lstlisting}
$ SSH -v git@github.com
				\end{lstlisting}
			\end{quote}
			返回
			\begin{quote}
				\begin{lstlisting}
$ Are you sure you want to continue connecting (yes/no)?
				\end{lstlisting}
			\end{quote}
			时,输入yes。
			最后,返回
			\begin{quote}
				\begin{lstlisting}
You've successfully authenticated, but GitHub does not provide shell access.
				\end{lstlisting}
			\end{quote}
			时,SSH密钥配置成功。
		\end{enumerate}
	% section 安装与配置 (end)
	\section{Git基本概念} % (fold)
	\label{sec:git基本概念}
		\subsection{仓库(Repositories)}
		仓库即为储存一个项目的代码的目录。仓库里包括了一个项目的全部代码和README文件。可以用本地现有的代码新建一个仓库,并上传至Git,也可以将Git上已有的仓库克隆(clone)到本地,进行操作。
		\par从本地新建一个仓库的操作是:
			\begin{quote}
				\begin{lstlisting}
$ cd \c\...PATH
$ git init
				\end{lstlisting}
			\end{quote}
		\par 接着,用{\tt git add}命令来添加改项目中需要追踪的文件(如一个TeX项目中,需要追踪的文件即为.tex文件,.pdf文件和其他的素材文件,而一些临时文件则不需要进行追踪)。
			\begin{quote}
				\begin{lstlisting}
$ git add *.* $ git add *.* $ git add *.* ......
				\end{lstlisting}
			\end{quote}
		\par 从现有的仓库克隆的方法为:
			\begin{quote}
				\begin{lstlisting}
$ git clone git://github.com/....../*.git
				\end{lstlisting}
			\end{quote}
		\par 现有的仓库的克隆地址可以在项目的Github页面上找到。这会在当前目录下创建一个名为“*”的目录,其中包含一个 .git 的目录文件,用于保存下载下来的所有版本记录,然后从中取出最新版本的文件拷贝。在仓库的目录下键入{\tt git status}命令,可以查看已跟踪和未跟踪的文件列表。

	% section git基本概念 (end)
\end{document}